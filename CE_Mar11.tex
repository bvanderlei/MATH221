\documentclass[11pt]{article}
\usepackage[letterpaper, margin=1in]{geometry}
\usepackage{amsmath, amssymb, graphicx, epsfig, fleqn}
\setlength{\parindent}{0pt}
\newcommand{\ud}{\,\mathrm{d}}
\everymath{\displaystyle}
\def\FillInBlank{\rule{2.5in}{.01in} }
\newcommand{\Ker}{\textrm{Ker}\,}
\newcommand{\Image}{\textrm{Im}\,}

\pagestyle{empty}

\begin{document}
\begin{center}
\Large
\rm{Math 221}
\\
\rm{Class Exercises:  Mar. 11}
\\
\end{center}
\vspace{0.2in}
\fboxsep0.5cm

\begin{enumerate}
	\item{ 
A transformation $T:\mathbb{P}_{2}\to\mathbb{P}_{2}$ is defined by $T(p(x))=p(x+1)$. 

\begin{enumerate}
	\item {Find $[T]_{\alpha}^{\alpha}$ if
\begin{displaymath}
\alpha = \left\{1, x, x^2\right\}.
\end{displaymath}	
}
	\item {Find $[T]_{\beta}^{\beta}$ if
	\begin{displaymath}
	\beta = \left\{1+x, 1-x, x+x^2 \right\}.
	\end{displaymath}	
}
\item{Draw a commutative diagram demonstrating your answers with $p(x)=x^2$.}
\end{enumerate}
}

\vspace{1in}

\item{Suppose $T:\mathbb{R}^3\to\mathbb{R}^3$ with 
	\begin{displaymath}
	T\left(\left[ \begin{array}{r} x_1 \\ x_2 \\ x_3 \end{array} \right] \right) =  \left[ \begin{array}{c} 3x_3 \\ x_1 \\ 2x_3 \end{array} \right] \quad \quad \quad 
	\textrm{and}
	\quad\quad\quad
	\alpha = \left\{
	\left[ \begin{array}{r} 1 \\ 0 \\ 0   \end{array}\right],
	\left[ \begin{array}{r} 0 \\ 1 \\ 0  \end{array}\right],
	\left[ \begin{array}{r} 0 \\ 0 \\ 1  \end{array}\right]
	\right\}
	\end{displaymath}
Find the basis $\beta$ such that $[T]_{\alpha}^{\beta} = I$.
}
\end{enumerate}

\end{document}


