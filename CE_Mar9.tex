\documentclass[11pt]{article}
\usepackage[letterpaper, margin=1in]{geometry}
\usepackage{amsmath, amssymb, graphicx, epsfig, fleqn}
\setlength{\parindent}{0pt}
\newcommand{\ud}{\,\mathrm{d}}
\everymath{\displaystyle}
\def\FillInBlank{\rule{2.5in}{.01in} }
\newcommand{\Ker}{\textrm{Ker}\,}
\newcommand{\Image}{\textrm{Im}\,}

\pagestyle{empty}

\begin{document}
\begin{center}
\Large
\rm{Math 221}
\\
\rm{Class Exercises:  Mar. 9}
\\
\end{center}
\vspace{0.2in}
\fboxsep0.5cm

A transformation $T:\mathbb{M}_{2\times 2}\to\mathbb{M}_{2\times 2}$ is defined by 

\begin{displaymath}
T\left(\left[ \begin{array}{cc} a & b \\ c & d \end{array}\right]\right) =   \left[ \begin{array}{cc} a & 0 \\ 3c & d \end{array}\right].
\end{displaymath}

\begin{enumerate}
	\item {Show that $T$ is linear.}
	\item {Find a basis for $\Image(T)$.}
	\item {Is $T$ an isomorphism?}
	\item {Find $[T]_{\alpha}^{\alpha}$ if
\begin{displaymath}
\alpha = \left\{
\left[ \begin{array}{cc} 1 & 0 \\ 0 & 0   \end{array}\right],
\left[ \begin{array}{cc} 0 & 1  \\ 0 & 0   \end{array}\right],
\left[ \begin{array}{cc} 0 & 0  \\ 1 & 0   \end{array}\right],
\left[ \begin{array}{cc} 0 & 0  \\ 0 & 1   \end{array}\right]
\right\}.
\end{displaymath}	
}
	\item {Find $[T]_{\beta}^{\alpha}$ if
	\begin{displaymath}
	\beta = \left\{
	\left[ \begin{array}{cc} 1 & 1 \\ 0 & 0   \end{array}\right],
	\left[ \begin{array}{cc} 0 & 0  \\ 1 & 1   \end{array}\right],
	\left[ \begin{array}{cc} 1 & 0  \\ 1 & 0   \end{array}\right],
	\left[ \begin{array}{cc} 0 & 1  \\ 0 & 1   \end{array}\right]
	\right\}.
	\end{displaymath}	
}
\end{enumerate}


\end{document}


