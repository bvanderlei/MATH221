\documentclass[11pt]{article}
\usepackage[letterpaper, margin=1in]{geometry}
\usepackage{amsmath, amssymb, graphicx, epsfig, fleqn}
\setlength{\parindent}{0pt}
\newcommand{\ud}{\,\mathrm{d}}
\everymath{\displaystyle}
\def\FillInBlank{\rule{2.5in}{.01in} }
\pagestyle{empty}

\begin{document}
\begin{center}
\Large
\rm{Math 221}
\\
\rm{Class Exercises:  Mar. 2}
\\
\end{center}
\vspace{0.2in}
\fboxsep0.5cm

A linear transformation $T:\mathbb{R}^3\to\mathbb{R}^2$ is defined by 

\begin{displaymath}
T\left(\left[ \begin{array}{r} x_1 \\ x_2 \\ x_3 \end{array}\right]\right) =  \left[ \begin{array}{r} x_1-2x_2 \\ x_1+3x_3 \end{array}\right].
\end{displaymath}

Show that $T$ is \textbf{surjective} (onto).


\vspace{4in}


If $S:\mathbb{R}^2\to\mathbb{R}^4$, explain why $S$ cannot be \textbf{onto}.

\pagebreak

A linear transformation $T:\mathbb{R}^3\to\mathbb{R}^3$ is defined by 

\begin{displaymath}
T\left(\left[ \begin{array}{r} x_1 \\ x_2 \\ x_3 \end{array}\right]\right) =  \left[ \begin{array}{c} 2x_2 \\ x_2 - x_3 \\ x_1+3x_3 \end{array}\right].
\end{displaymath}

Show that $T$ is \textbf{injective} (one-to-one).

\pagebreak
%\vspace{0.3in}

Suppose $T:\mathbb{R}^3\to\mathbb{R}^4$ and $S:\mathbb{R}^4\to\mathbb{R}^2$ are defined as follows.

\begin{displaymath}
T\left(\left[ \begin{array}{r} x_1 \\ x_2 \\ x_3 \end{array}\right]\right) = 
\left[ \begin{array}{c} x_1+x_2 \\ x_1 \\ x_2-x_3 \\ x_1-x_2 \end{array}\right]
\quad\quad\textrm{and}\quad\quad
S\left(\left[ \begin{array}{r} x_1 \\ x_2 \\ x_3 \\ x_4 \end{array}\right]\right) = 
\left[ \begin{array}{c} x_1-x_2+x_4 \\ x_2+3x_4 \end{array}\right]
\end{displaymath}

\begin{enumerate}
	\item {Find a matrix $B$ so that $S\circ T:\mathbb{R}^3\to\mathbb{R}^2$ is defined by $(S\circ T)(x) = Bx$.}
	\item {Is $(S\circ T)$ one-to-one?  Explain.}
\end{enumerate}




\end{document}


