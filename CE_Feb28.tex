\documentclass[11pt]{article}
\usepackage[letterpaper, margin=1in]{geometry}
\usepackage{amsmath, amssymb, graphicx, epsfig, fleqn}
\setlength{\parindent}{0pt}
\newcommand{\ud}{\,\mathrm{d}}
\everymath{\displaystyle}
\def\FillInBlank{\rule{2.5in}{.01in} }
\pagestyle{empty}

\begin{document}
\begin{center}
\Large
\rm{Math 221}
\\
\rm{Class Exercises:  Feb. 28}
\\
\end{center}
\vspace{0.2in}
\fboxsep0.5cm

Give a formula for a linear transformation $T:\mathbb{R}^3\to\mathbb{R}^2$ such that

\begin{displaymath}
T\left(\left[ \begin{array}{r} 2 \\ 2 \\ -1 \end{array}\right]\right) = 
\left[ \begin{array}{r} 7 \\ 1 \end{array}\right]
\end{displaymath}

\begin{enumerate}
	\item {Can you show that the transformation defined by your formula is linear?}
	\item {Can you find a matrix $A$ so that $T(x)=Ax$?}
\end{enumerate}

\pagebreak

Give a geometric description of the transformation $S:\mathbb{R}^2\to\mathbb{R}^2$ defined by

\begin{displaymath}
S\left(\left[ \begin{array}{r} x_1 \\ x_2 \end{array}\right]\right) = 
\left[ \begin{array}{rr} 0 & -1 \\ 1 & 0 \end{array}\right]
\left[ \begin{array}{r} x_1 \\ x_2 \end{array}\right]
\end{displaymath}


\vspace{4in}

Give a geometric description of the transformation $S:\mathbb{R}^2\to\mathbb{R}^2$ defined by

\begin{displaymath}
S\left(\left[ \begin{array}{r} x_1 \\ x_2 \end{array}\right]\right) = 
\left[ \begin{array}{rr} 1 & 1 \\ -1 & 1 \end{array}\right]
\left[ \begin{array}{r} x_1 \\ x_2 \end{array}\right]
\end{displaymath}

\pagebreak
%\vspace{0.3in}

Suppose $T:\mathbb{R}^2\to\mathbb{R}^4$ and

\begin{displaymath}
T\left(\left[ \begin{array}{r} 3 \\ 2 \end{array}\right]\right) = 
\left[ \begin{array}{r} 0 \\ -1 \\ 2 \\ -1 \end{array}\right]
\quad\quad\textrm{and}\quad\quad
T\left(\left[ \begin{array}{r} -1 \\ 2 \end{array}\right]\right) = 
\left[ \begin{array}{r} 1 \\ 0 \\ 0 \\ 1 \end{array}\right]
\end{displaymath}

\vspace{0.2in}
\begin{enumerate}
	\item {Calculate 
		\begin{displaymath}
		T\left(\left[ \begin{array}{r} 1 \\ 2 \end{array}\right]\right)
		\end{displaymath}
	}
	\item {Calculate 
		\begin{displaymath}
		T\left(\left[ \begin{array}{r} x_1 \\ x_2 \end{array}\right]\right)
		\end{displaymath}
	}
\end{enumerate}




\end{document}


